\documentclass[12pt,a4paper,oneside]{book}
\usepackage[italian]{babel}
\usepackage[UTF8]{inputenc}
%\usepackage[latin1]{inputenc}
\usepackage{amsmath}
\usepackage{amsthm}
\newtheorem{thm}{Teorema}
\usepackage{amsfonts}
\usepackage{amssymb}
\usepackage{graphicx}
\usepackage{eso-pic}
\usepackage{setspace}
\usepackage{fancyhdr} 
\newcommand{\fncyblank}{\fancyhf{}}
\newenvironment{abstract}% 
{\cleardoublepage\fncyblank\null \vfill\begin{center}% 
\bfseries \abstractname \end{center}}% 
{\vfill\null}
\usepackage{afterpage}
\newcommand\blankpage{%
	\null
	\thispagestyle{empty}%
	\addtocounter{page}{0}%
	\newpage}
\newcommand\AlCentroPagina[1]{% 
\AddToShipoutPicture*{\AtPageCenter{%
\makebox(0,0){\includegraphics%
[width=1\paperwidth]{#1}}}}}
\setlength{\parindent}{0pt}
\setlength{\parskip}{1ex plus 0.5ex minus 0.2ex}
\author{Maria Luisa Feola}
\title{Introduzione}
\begin{document}


\chapter*{Introduzione}
\addcontentsline{toc}{chapter}{Introduzione}

Il presente lavoro ha come oggetto le reti neurali artificiali ed in particolar modo il funzionamento delle reti di Hopfield.\\
Il primo capitolo si apre con una breve presentazione storica sulle origini delle reti neurali artificiali e partendo da una breve esibizione del neurone biologico e della sua attività, sono mostrate le analogie che quest'ultimo ha con le reti artificiali. \\
Costituite da un insieme di interconnessioni e neuroni artificiali, tali reti sono realizzate ispirandosi ai meccanismi ed alle strutture delle reti neurali biologiche ed il loro obiettivo è la riproduzione delle attività tipiche del cervello umano. Pertanto in tale capitolo è illustrato il modello matematico del neurone artificiale, nonché il processo di attivazione di un generico neurone, esibendo le funzioni di attivazione comunemente utilizzate e le principali architetture attraverso cui possono essere modellate le reti artificiali.\\
Nel secondo capitolo è dedicato ampio spazio alle fasi di apprendimento delle reti artificiali ed ai paradigmi attraverso i quali una rete può apprendere, sottolineando il tratto caratteristico del sistema nervoso a cui tali reti neurali si rifanno, ovvero la capacità di acquisire esperienza da esempi del mondo reale. Dunque è contemplata la fase di addestramento di tali strutture e sono illustrati i maggiori processi di apprendimento e le attività che una rete può svolgere in base alla tipologia del problema che deve risolvere.\\
Infine è presentato un esempio di apprendimento supervisionato attraverso il percettrone elementare, proposto da Frank Rosenblatt nel 1958 e che risulta essere ancora oggi la forma più semplice di rete neurale.\\
Il terzo capitolo è incentrato sulle reti di Hopfield, una tipologia di rete neurale artificiale nota per il suo utilizzo nella costruzione di memorie associative. La rete può essere intesa come un sistema dinamico non lineare che evolve verso una configurazione stabile. Tale stabilità risulta essere strettamente legata ad una funzione interna che rappresenta l'energia totale del sistema e pertanto sono riportati cenni sul concetto di stabilità di Lyapunov e di funzione di Lyapunov.\\
Il capitolo si conclude con l'esibizione di una rete di Hopfield nel caso discreto e nel caso continuo e con delle osservazioni che concernono i limiti inevitabili legati a questa struttura.
\end{document}
